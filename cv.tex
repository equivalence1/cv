\documentclass[10pt,a4paper,sans]{moderncv} % Font sizes: 10, 11, or 12; paper sizes: a4paper, letterpaper, a5paper, legalpaper, executivepaper or landscape; font families: sans or roman

\moderncvstyle{classic} % CV theme - options include: 'casual' (default), 'classic', 'oldstyle' and 'banking'
\moderncvcolor{blue} % CV color - options include: 'blue' (default), 'orange', 'green', 'red', 'purple', 'grey' and 'black'

\usepackage{lipsum} % Used for inserting dummy 'Lorem ipsum' text into the template

\usepackage[scale=0.89, top=1.5cm, bottom=0cm]{geometry} % Reduce document margins

\setlength{\hintscolumnwidth}{3cm} % Uncomment to change the width of the dates column
%\setlength{\makecvtitlenamewidth}{10cm} % For the 'classic' style, uncomment to adjust the width of the space allocated to your name

%----------------------------------------------------------------------------------------
%	NAME AND CONTACT INFORMATION SECTION
%----------------------------------------------------------------------------------------

\firstname{Dmitrii} % Your first name
\familyname{Kravchenko} % Your last name

% All information in this block is optional, comment out any lines you don't need
\title{Curriculum Vitae}
%\address{Veselnaya St., 9-15}{Saint Petersburg, Russia}
\mobile{ +7 (950) 0472765}
\email{equivalence1@gmail.com}
\github{Github: \href{https://github.com/equivalence1/}{\color{blue}{equivalence1}}}
%\codeforces{Codeforces: \href{http://codeforces.com/profile/Nikitosh/}{\color{blue}{Nikitosh}}}

%----------------------------------------------------------------------------------------

\begin{document}

\makecvtitle % Print the CV title

%----------------------------------------------------------------------------------------
%	EDUCATION SECTION
%----------------------------------------------------------------------------------------

\section{Education}

%\cventry{2009--2014}{School student}{Saint Petersburg Presidential Physics and Mathematics Lyceum 239}{Saint Petersburg}{}{} 
%\cventry{2011, 2012, 2013}{Student}{Summer Informatics School}{Kostroma}{}{} 
\cventry{2014--present}{Bachelor of Computer Science}{St. Petersburg Academic University of Russian Academy of Sciences, Department of Mathematics and Information Technology}{Saint Petersburg}{}{}
\cventry{2018--2020}{Master of Computer Science}{Higher School of Economics in Saint-Petersburg, School of Data Analysis and Artificial Intelligence}{Saint Petersburg}{}{}


\section{Theoretical background}

\cvitem{Programming}{Algorithms and data structures, Networks, Databases, Parallel Programming, Big Data, Software design}
\cvitem{Math}{Calculus, Algebra, Discrete mathematics, Probability theory, Mathematical logic, Information theory}

%----------------------------------------------------------------------------------------
%	WORK EXPERIENCE SECTION
%----------------------------------------------------------------------------------------
\section{Experience}

\cventry{2017-2018}{Software Engineer Intern}{Acronis}{St. Petersburg (remote)}{}
{Acronis has its own async io library written on C. The central part of it are coroutines. The
main flaw of this library for now is that it is single-threaded (i.e. all corouties are scheduled on the same kernel-level thread).
I expand the functionality by making it multi-threaded so we can use the full potential of a modern hardware.}

\cventry{2016}{Software Engineer Intern}{JetBrains}{St. Petersburg}{}
{Created "attach to local process" debugger feature in RubyMine IDE. It allows you to 
debug local process which wasn't initially started under debug mode. After you had finished your
debug session, you can detach from the process and it will continue to run as if nothing
had happened. This feature was successfully released and you can see more info \href{https://www.jetbrains.com/help/ruby/attaching-to-local-process.html}{\color{blue}{here}}.}

\section{Teaching experience}

\cventry{2017}{Teaching assistant}{Saint Petersburg Academic University}{Saint Petersburg}{}{Assessed students' homeworks for "Operating systems" and "Computer Architecture" courses.}

\section{Projects}

\cventrynotbold{2017}{\href{https://github.com/equivalence1/aucont}{\color{blue}{Linux containers}}}{}{C, Python}{}
{My own implementation of linux containers. Basically, it is like Docker. Of course my version lacks some of docker's fancy features like Dockerhub and Dockerfile, but it's also more lightweight.
It can start and stop containers, list all currently running containers, execute a command inside of a container. 
Mostly written on C but some parts are implemented using Python.}

\cventrynotbold{2016}{\href{https://github.com/equivalence1/p2p_chat}{\color{blue}{Peer-to-peer chat}}}{}{Java}{\texttt{Maven, JavaFX, Google protobuf, gRPC}}
{Simple online peer-to-peer chat. Written completely on Java. I use JavaFX library for UI, Google protobuf for message transmission, gRPC to show that the other user is currently typing.}

\cventrynotbold{2015--2016}{\href{https://github.com/auTimetable/auTimetable}{\color{blue}{auTimetable}}}{}{Java}{\texttt{Android, Google app engine, Twitter Bootstrap}}
{Android application to manage and view students' schedule of the classes. Application itself is written on Java. Server part
is written on Java with a use of Google app engine for the back-end and JavaScript with a use of Twitter Bootstrap for the front-end. Web application allows entitled (i.e. teachers) users to alter students'
schedule, while android part only allows to view the timetable.}

%----------------------------------------------------------------------------------------
%	ACHIEVEMENTS SECTION
%----------------------------------------------------------------------------------------

\section{Achievements}

\cvitemwithcomment{2014}{All-Russian Olympiad of School Students in Informatics, 29th place}{Programming competition among the best high school students in Russia}
\cvitemwithcomment{2012}{School All-Russian Team Programming Championship, 12st place}{Programming team competition among the best high school students in Russia}

%----------------------------------------------------------------------------------------
%	COMPUTER SKILLS SECTION
%----------------------------------------------------------------------------------------

\section{Technology summary}

\cvitem{\scriptsize{Languages (good knowledge)}}{Java, C}
\cvitem{\scriptsize{Languages (basic knowledge)}}{Go, C++, Python, Ruby, Scala, Kotlin, Haskell, JavaScript}
\cvitem{\scriptsize{Technologies}}{git, docker, postgreSQL, Linux, OpenCL, google protobuf, ROS, ElasticSearch}

%----------------------------------------------------------------------------------------
%	LANGUAGES SECTION
%----------------------------------------------------------------------------------------

\section{Languages}

\cvitemwithcomment{Russian}{Native speaker}{}
\cvitemwithcomment{English}{Upper Intermediate}{}

\end{document}
