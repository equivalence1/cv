\documentclass[10pt,a4paper,sans]{moderncv} % Font sizes: 10, 11, or 12; paper sizes: a4paper, letterpaper, a5paper, legalpaper, executivepaper or landscape; font families: sans or roman

\moderncvstyle{classic} % CV theme - options include: 'casual' (default), 'classic', 'oldstyle' and 'banking'
\moderncvcolor{blue} % CV color - options include: 'blue' (default), 'orange', 'green', 'red', 'purple', 'grey' and 'black'

\usepackage{lipsum} % Used for inserting dummy 'Lorem ipsum' text into the template

\usepackage[scale=0.89, top=1.5cm, bottom=0cm]{geometry} % Reduce document margins

\setlength{\hintscolumnwidth}{3cm} % Uncomment to change the width of the dates column
%\setlength{\makecvtitlenamewidth}{10cm} % For the 'classic' style, uncomment to adjust the width of the space allocated to your name

%----------------------------------------------------------------------------------------
%	NAME AND CONTACT INFORMATION SECTION
%----------------------------------------------------------------------------------------

\firstname{Dmitrii} % Your first name
\familyname{Kravchenko} % Your last name

% All information in this block is optional, comment out any lines you don't need
\title{Curriculum Vitae}
%\address{Veselnaya St., 9-15}{Saint Petersburg, Russia}
%\mobile{ +7 (950) 0472765}
\email{equivalence1@gmail.com}
\github{Github: \href{https://github.com/equivalence1/}{\color{blue}{equivalence1}}}
%\codeforces{Codeforces: \href{http://codeforces.com/profile/Nikitosh/}{\color{blue}{Nikitosh}}}

%----------------------------------------------------------------------------------------

\begin{document}

\makecvtitle % Print the CV title

%----------------------------------------------------------------------------------------
%	EDUCATION SECTION
%----------------------------------------------------------------------------------------

\section{Education}

%\cventry{2009--2014}{School student}{Saint Petersburg Presidential Physics and Mathematics Lyceum 239}{Saint Petersburg}{}{} 
%\cventry{2011, 2012, 2013}{Student}{Summer Informatics School}{Kostroma}{}{} 
\cventry{2014--2018}{Bachelor of Computer Science}{St. Petersburg Academic University of Russian Academy of Sciences, Department of Mathematics and Information Technology}{Saint Petersburg}{}{}
\cventry{2018--2020}{Master of Computer Science}{Higher School of Economics in Saint-Petersburg, School of Data Analysis and Artificial Intelligence}{Saint Petersburg}{}{}
\cventry{2020--}{CS PhD}{Higher School of Economics in Saint-Petersburg, School of Data Analysis and Artificial Intelligence}{Saint Petersburg}{}{}

%----------------------------------------------------------------------------------------
%	WORK EXPERIENCE SECTION
%----------------------------------------------------------------------------------------
\section{Experience}

\cventry{2019-present}{Full-time Software Engineer}{Yandex}{St. Petersburg}{}
{Being part of Yandex.Cloud Machine Learning tools team, my job was to provide users (we had both B2U and B2B settings) with such ML tools in cloud as ASR/TTS, Translation, Vision and many more. I personally was mostly working with our ASR product, Yandex.SpeechKit, working on both its infrastructure and ML models.}

\cventry{2018-2018}{Software Engineering Intern}{afi.ai}{St. Petersburg (remote)}{}
{Afi.ai makes corporate backups for products like GSuite, MS Office 365. My task was to implement indexing of those backups in ElasticSearch cluster so that users could search through their documents. The system ought to be horizontally scalable, multitenant, fast and secure.}

\cventry{2017-2018}{Software Engineering Intern}{Acronis}{St. Petersburg (remote)}{}
{Acronis has its own async I/O library written in C. The central part of it are coroutines. The
main flaw of this library at the beginning of my internship was that it was single-threaded (i.e. all corouties were scheduled on the same kernel-level thread).
I was implementing multi-threaded coroutines scheduler for our I/O library, so it can use the full potential of a modern hardware.}

\cventry{2016}{Software Engineering Intern}{JetBrains}{St. Petersburg}{}
{Created "attach to local process" debugger feature in RubyMine IDE, which allows one to
debug a local process which wasn't initially started under debug mode.}

\section{Teaching experience}

\cventry{2017}{Teaching assistant}{Saint Petersburg Academic University}{Saint Petersburg}{}{Assessed students' homeworks for "Operating systems" and "Computer Architecture" courses.}

\section{Projects}

\cventry{2018-present}{\href{https://github.com/Noxoomo/ml_lib}{\color{blue}{NN + GBDT}}}{}{C++, libtorch, catboost}{}
{In this project I, my peers, and my advisor are trying to combine two seemingly incompatible models -- Artificial Neural Networks and Gradient Boosted Decision Trees. We came up with an idea of how they can be jointly trained, and our test experiments show that we can achive good quality on image classification tasks. We continue our work and have a lot of ideas to try.}

\cventry{2017}{\href{https://github.com/equivalence1/aucont}{\color{blue}{Linux containers}}}{}{C, Python}{}
{My own implementation of linux containers. Basically, it is like Docker. Of course my version lacks some of docker's fancy features like Dockerhub and Dockerfile, but it's also more lightweight.
It can start and stop containers, list all currently running containers, execute a command inside of a container. 
Mostly written on C but some parts are implemented using Python.}

%----------------------------------------------------------------------------------------
%	ACHIEVEMENTS SECTION
%----------------------------------------------------------------------------------------

\section{Achievements}

\cvitemwithcomment{2014}{All-Russian Olympiad of School Students in Informatics, 29th place}{Programming competition among the best high school students in Russia}
\cvitemwithcomment{2012}{School All-Russian Team Programming Championship, 12st place}{Programming team competition among the best high school students in Russia}

%----------------------------------------------------------------------------------------
%	COMPUTER SKILLS SECTION
%----------------------------------------------------------------------------------------

\section{Technology summary}

\cvitem{\scriptsize{Languages}}{C/C++, Golang, Python, Java}
\cvitem{\scriptsize{Technologies}}{git, docker, SQL, Pytorch, Unix, OpenCL, gRPC/protobuf, ROS, ElasticSearch}


\end{document}
